\documentclass{article}
\usepackage[utf8]{inputenc}
\usepackage{amsmath} % For math environments
\usepackage{amsfonts} % For math fonts
\usepackage{amssymb} % For math symbols
\usepackage{graphicx} % To include graphics
\usepackage{hyperref} % For hyperlinks
\usepackage[margin=1in]{geometry} % Page margins
\usepackage{parskip} % Paragraph spacing instead of indent
\usepackage{url} % For typesetting URLs

\title{Section 4B - Extended Model: Advantage from Isolated Spot Markets and Synthetic Asset Solution}
% \author{}
\date{April 2024}

\begin{document}
% \maketitle % Uncomment to display title

\section{Section 4B - Extended Model: Advantage from Isolated Spot Markets and Synthetic Asset Solution}
\subsection{Step 1: Formalizing Isolated Spot Markets}

An isolated spot market can be defined mathematically as a one-dimensional price curve, $P(t)$:

Let the token's price at launch ($t_0$) be $P(t_0)$.

Each subsequent buy action directly increases the price by the amount of impact, $N$, proportional to the demand volume $Q$.

The isolated price function is thus:
\[
P(t_0 + n) = P(t_0) + \sum_{i=1}^{n} N(Q_i)
\]
Where:
\begin{itemize}
    \item $N(Q_i)$ is the price impact of buy transaction $i$ at time $t_i$
    \item Note: 'n' here represents the *number of transactions*, distinct from the time delay 'N' used in other models.
\end{itemize}

\subsection*{Consequences of Isolation}

\subsubsection*{Buy-first Advantage:}
\begin{itemize}
    \item Investors at $t_0$ buy at minimal price $P(t_0)$
    \item Investors at $t_0 + n$ must pay inflated price $P(t_0 + n)$
\end{itemize}

\subsubsection*{Sell Constraint & Inelastic Supply:}
\begin{itemize}
    \item Selling is only possible after buying. Initial supply $S(P)$ is extremely inelastic, consisting only of early buyers potentially taking profit.
    \item This allows demand $D(P)$ shocks (like initial insider buying) to cause excessive price increases.
\end{itemize}

Formally, investors who buy at $t_0 + n$ have permanently higher cost bases:
\[
P_{\text{basis,early}} = P(t_0)
\]
\[
P_{\text{basis,late}} = P(t_0 + n) > P(t_0)
\]
Thus, insiders maintain a perpetual pricing advantage, leading to ongoing asymmetric profits.

\section{Step 2: Illustrating Structural Insider Advantage}

Mathematically, early insiders' perpetual advantage ($A_a$) at any future time $t_m > t_0 + n$ is:
\[
A_a(t_m) = P(t_m) - P(t_0), \quad \text{always positive if market persists}
\]
Latecomers' perpetual disadvantage ($D_o$):
\[
D_o(t_m) = P(t_m) - P(t_0 + n), \quad \text{generally smaller or negative}
\]
Because:
\begin{itemize}
    \item $P(t_0 + n)$ is structurally inflated due to inelastic initial supply.
    \item Insiders continually extract profit by selling to latecomers at higher prices, potentially reaching an artificially high $P_{\text{max\_isolated}}$.
\end{itemize}

\section{Step 3: Introducing Synthetic Assets & Elastic Supply}

To restore equilibrium, introduce synthetic assets (derivatives like futures, options) allowing market participants to effectively short the asset without prior ownership.
\begin{itemize}
    \item Formally, this creates a synthetic supply curve $S_{\text{synthetic}}(P)$ that is available much earlier and is more elastic than the spot-only supply.
    \item Participants can now express negative price expectations by initiating short positions ($Q_{\text{short}}$) via derivatives.
    \item Define the synthetic asset price $P_{\text{synthetic}}(t)$ linked to the spot price $P_{\text{spot}}(t)$, potentially with a basis difference.
\end{itemize}

\textbf{Consequences:}
\begin{enumerate}
    \item \textbf{Increased Supply Elasticity:} The total supply $S_{\text{total}}(P) = S_{\text{spot}}(P) + S_{\text{synthetic}}(P)$ becomes significantly more responsive to price changes, especially at inflated levels.
    \item \textbf{Two-Sided Market:} Transforms the market from buy-dominated to two-sided, allowing for price discovery based on both positive and negative sentiment.
    \item \textbf{Outsider Counter-Strategy:} Outsiders arriving at $t_0 + N$ can now potentially profit by shorting $P_{\text{synthetic}}$ if they perceive the price driven up by insiders as overvalued, providing a mechanism for $G_o > 0$ even if they missed the initial entry.
\end{enumerate}

\section{Step 4: Mathematical Equilibrium Restoration & Insider Gain Capping}

The equilibrium price $P_{\text{equilibrium}}$ is determined where total demand equals total supply: $D(P) = S_{\text{total}}(P)$.

\subsection*{Price Equation with Synthetics:}
\[
P_{\text{equilibrium}}(t) = P(t_0) + \sum_{i} N(Q_{\text{buy},i}) + \sum_{j} N(Q_{\text{short},j})
\]
\begin{itemize}
    \item The negative price impact $N(Q_{\text{short},j})$ from synthetic short-selling directly counteracts the positive impact from buying.
    \item This leads $P_{\text{equilibrium}}(t)$ to converge towards $E[P(t)]$ (fair value) more rapidly and reliably than $P_{\text{spot}}(t)$ in an isolated market.
\end{itemize}

\subsection*{Capping Insider Gains ($G_a$):}
\begin{itemize}
    \item In an isolated market, insiders could potentially drive the price to $P_{\text{max\_isolated}}$ before facing significant selling pressure.
    \item With synthetics, potential short-sellers anticipating a reversion provide selling pressure much earlier.
    \item This creates a lower effective price ceiling, $P_{\text{max\_synthetic}} < P_{\text{max\_isolated}}$.
    \item Insider gain $G_a = Q_a \cdot [E[P_{\text{sell}}] - P(t_0+\delta)]$ is reduced because $E[P_{\text{sell}}] \le P_{\text{max\_synthetic}}$.
\end{itemize}

\subsection*{Neutralizing Perpetual Advantage:}
\begin{itemize}
    \item As $Q_{\text{short}}$ approaches $-Q_{\text{buy}}$ (balancing flows), the upward price distortion diminishes.
    \item $\lim (S_{\text{synthetic}} \to \infty \text{ at } P > E[P]) [P_{\text{equilibrium}}(t) - E[P(t)]] \to 0$
    \item The perpetual insider advantage $A_a(t_m) = P_{\text{equilibrium}}(t_m) - P(t_0)$ is driven towards reflecting only the true change in expected value, not the structural advantage from early entry into an illiquid market.
    \item $\lim_{Q_{\text{short}} \to -Q_{\text{buy}}} A_a(t_m) \text{ based purely on timing} \to 0$
\end{itemize}
This restores balance by neutralizing insiders' exclusive price advantage derived from market isolation.

\section{Step 5: Illustrative Example}

\subsection*{Without Synthetic Assets:}
\begin{itemize}
    \item Insiders buy at \$0.01. Inelastic supply allows price to inflate easily.
    \item Price driven to \$0.10 ($P_{\text{max\_isolated}}$ example) by follow-on buying.
    \item Outsiders buy high, insiders sell near the peak.
    \item Perpetual insider advantage: Cost basis \$0.01 vs outsiders at \$0.10.
\end{itemize}

\subsection*{With Synthetic Assets:}
\begin{itemize}
    \item As price rises above perceived fair value (e.g., \$0.05), outsiders/arbitrageurs short-sell synthetics at $P_{\text{synthetic}} \approx \$0.10$.
    \item This selling pressure prevents the price from staying excessively high ($P_{\text{max\_synthetic}} < \$0.10$).
    \item Equilibrium price may settle closer to \$0.05$.
    \item Insiders' gains are capped, and outsiders have opportunities to profit from shorting.
\end{itemize}
Thus, synthetic derivatives mathematically dampen manipulation by enabling elastic, two-sided liquidity.

\section{Step 6: Generalized Conclusion (Mathematically Demonstrated)}

\subsection*{In isolated spot markets:}
\begin{itemize}
    \item Structural insider advantage arises from price isolation and inelastic initial supply.
\end{itemize}

\subsection*{Synthetic assets mathematically restore equilibrium by:}
\begin{itemize}
    \item Introducing elastic (synthetic) supply via short-selling.
    \item Facilitating balanced two-sided trading.
    \item Capping artificial price peaks and reducing insider gains.
    \item Driving prices toward a fair equilibrium, neutralizing timing-based advantages.
\end{itemize}

The introduction of derivative-based synthetic assets transforms isolated spot markets into more balanced, fair, and efficient trading environments.

\end{document} 