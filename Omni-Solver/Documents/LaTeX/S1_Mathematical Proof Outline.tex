\documentclass{article}
\usepackage[utf8]{inputenc}
\usepackage{amsmath} % For math environments
\usepackage{amsfonts} % For math fonts
\usepackage{amssymb} % For math symbols
\usepackage{graphicx} % To include graphics
\usepackage{hyperref} % For hyperlinks
\usepackage[margin=1in]{geometry} % Page margins
\usepackage{parskip} % Paragraph spacing instead of indent
\usepackage{url} % For typesetting URLs
\usepackage{tabularx} % For tables
\usepackage{booktabs} % For better table lines

\title{Section 1 - Mathematical Proof Outline}
% \author{}
\date{April 2024}

\begin{document}
% \maketitle % Uncomment to display title

\section{Step 1: Define Market Participants and Information Sets}

\subsection*{Participants:}
\begin{itemize}
    \item \textbf{Insiders (I)}: Possess advanced information at time $t_0$
    \item \textbf{Outsiders (O)}: Receive information at time $t_0 + N$
\end{itemize}

\subsection*{Information Set:}
\begin{itemize}
    \item Let $I(t)$ represent information available at time $t$
    \item Insiders have $I(t_0)$, while outsiders only have $I(t_0 + N)$
\end{itemize}

\section{Step 2: Information Asymmetry Opportunity (Economic Model)}

Consider a simplified asset price model, where the price at any time $t$ is:
\[
P(t) = P(t_0) + \alpha \cdot I(t) + \epsilon_t
\]
Where:
\begin{itemize}
    \item $P(t_0)$ is the initial price
    \item $\alpha$ measures the price impact of information
    \item $\epsilon_t$ is a random noise component, $\epsilon_t \sim \mathcal{N}(0, \sigma^2)$
\end{itemize}

\subsection*{Economical Gain for Insiders ($G_a$):}
Insiders trade at $t_0$:
\[
G_a = E[P(t) - P(t_0) | I(t_0)] = \alpha \cdot I(t_0)
\]

\subsection*{Economical Gain for Outsiders ($G_o$):}
Outsiders trade at $t_0 + N$:
\[
G_o = E[P(t') - P(t) | I(t)], \quad t' > t > t_0
\]
Since by definition, $I(t_0)$ is superior (earlier), we have:
\[
G_a > G_o
\]
Thus, insiders always have strictly greater expected gains due to information asymmetry.

\section{Step 3: Game-Theoretical Nash Equilibrium}

Model the game with two strategies:
\begin{itemize}
    \item \textbf{Ethical (E)}: Transparent & fair
    \item \textbf{Unethical (U)}: Exploitative, maximizing personal gain
\end{itemize}

\subsection*{Payoff matrices ($\pi$) for Founders (F):}

\begin{tabularx}{\linewidth}{@{}lXX@{}}
\toprule
Founder   & Insider Behavior               & Outsider Behavior                \
\midrule
Ethical   & Low payoff                     & Low payoff                       \
Unethical & High payoff (early exploitation) & Medium payoff (later market entry) \
\bottomrule
\end{tabularx}

\subsection*{Rationality assumption (utility maximization):}
\[
U_F(U) > U_F(E)
\]
Due to early access and information asymmetry.

Thus, Nash Equilibrium favors unethical behavior:

Ethical strategy is strictly dominated by unethical strategy for insiders, given isolated market structure:
\[
U_F(U) \succ U_F(E)
\]

\section{Step 4: Structural Incentive for Unethical Behavior}

Isolated spot markets allow infinite, zero-cost market creation ($C_{market} = 0$):

Let expected profit from market creation be $G_a$:

\subsection*{Insiders:}
\[
G_a = \alpha \cdot I(t_0) - C_{market} > 0
\]

\subsection*{Outsiders:}
\[
G_o = \alpha \cdot I(t_0 + N) - C_{market} \quad (t > t_0)
\]
Given $C_{market} = 0$ and $I(t_0) > I(t_0 + N)$, insiders consistently generate profit with no cost or risk:
\[
G_a \gg G_o
\]

\section{Step 5: Impact of Financial Instrument ("Omni-Solver")}

Introduce an Omni-Solver to price perpetual derivatives fairly:

\subsection*{Fair value price $F(t)$:}
\[
F(t) = P(t) + \beta \cdot V(t)
\]
where:
\begin{itemize}
    \item $V(t)$ is the fair-value premium derived from continuous open market participation
    \item $\beta$ normalizes information asymmetry premium
\end{itemize}

Omni-Solver eliminates isolated information asymmetry opportunities ($G_a \approx G_o$):
\[
E[F(t_0) | I(t_0)] \approx E[F(t) | I(t)], \quad t \ge t_0
\]
This theoretically neutralizes insiders' asymmetric advantages, enforcing market fairness.

\section*{Conclusion (Mathematical Justification)}

Mathematically demonstrate that isolated spot markets structurally incentivize unethical insider behavior by formally showing:

\begin{enumerate}
    \item \textbf{Information Asymmetry}: Earlier market participation yields greater economical gains ($G_a > G_o$)
    \item \textbf{Game-Theoretic Dominance}: Unethical behavior is rationally dominant due to immediate financial incentives
    \item \textbf{Zero-Entry Cost Markets}: Incentivize repeated unethical behavior through minimal barriers
    \item \textbf{Fairness via Financial Innovation}: The introduction of a tool like Omni-Solver theoretically equalizes expected returns and removes asymmetric insider advantages.
\end{enumerate}

\end{document} 