\documentclass{article}
\usepackage[utf8]{inputenc}
\usepackage{amsmath} % For math environments
\usepackage{amsfonts} % For math fonts
\usepackage{amssymb} % For math symbols
\usepackage{graphicx} % To include graphics
\usepackage{hyperref} % For hyperlinks
\usepackage[margin=1in]{geometry} % Page margins
\usepackage{parskip} % Paragraph spacing instead of indent
\usepackage{url} % For typesetting URLs

\title{Section 4A - Mathematical Derivation of Insider Advantage and Market Equilibrium}
% \author{}
\date{April 2024}

\begin{document}
% \maketitle % Uncomment to display title

\section{Section 4A - Mathematical Derivation of Insider Advantage and Market Equilibrium}

\subsection{Formal Definition of Isolated Spot Market}

Consider an isolated spot market defined by a one-dimensional price curve $P(t)$. At token launch $t_0$, the price is defined as:
\[
P(t_0 + n) = P(t_0) + \sum_{i=1}^{n} N(Q_i)
\]
where:
\begin{itemize}
    \item $P(t_0)$ is the initial price
    \item $N(Q_i)$ represents the price impact of the $i^{th}$ buy action at time $t_i$
\end{itemize}

This formula demonstrates the direct additive impact of each subsequent purchase, reflecting the isolated and unidirectional nature of trading.

\section{Insider Structural Advantage}

Early insiders purchasing at $t_0$ incur a significantly lower cost basis compared to outsiders who buy at later times $t_0 + n$. The perpetual insider advantage $A_a$ can thus be represented as:
\[
A_a(t_m) = P(t_m) - P(t_0), \quad t_m > t_0 + n
\]
Conversely, outsiders face a perpetual structural disadvantage $D_o$:
\[
D_o(t_m) = P(t_m) - P(t_0 + n), \quad \text{where typically } D_o(t_m) < A_a(t_m)
\]
This explicitly quantifies insiders' lasting advantage resulting directly from early entry.

\section{Introducing Synthetic Assets to Restore Equilibrium}

To mitigate insiders' structural advantage, introduce synthetic assets allowing outsiders to short-sell. These synthetic assets follow a price closely aligned to the underlying isolated spot market price $P_{spot}(t)$. Define the synthetic price as:
\[
P_{synthetic}(t) \approx P_{spot}(t)
\]
Introducing short-selling (negative liquidity $Q_{short} < 0$) broadens the price curve:
\[
P_{equilibrium}(t) = P(t_0) + \sum_{i=1}^{n} N(Q_i) + \sum_{j=1}^{m} N(Q_{short,j}), \quad Q_{short,j} < 0
\]
This formulation mathematically demonstrates how negative liquidity from synthetic assets offsets positive buy-side pressures, driving the market towards equilibrium.

\section{Equilibrium Condition and Insider Advantage Mitigation}

As short-selling through synthetic assets balances buying pressure:
\[
\lim_{Q_{short} \to -Q_{buy}} A_a(t_m) \to 0
\]
Thus, the synthetic assets systematically diminish insiders' advantage, eliminating perpetual asymmetry.

\section{Information Asymmetry and Arbitrage}

Define insiders (I) who possess information at $t_0$ and outsiders (O) who receive delayed information at $t_0 + N$. The asset price $P(t)$ is modeled as:
\[
P(t) = P(t_0) + \alpha \cdot I(t) + \epsilon_t, \quad \epsilon_t \sim \mathcal{N}(0, \sigma^2)
\]
Arbitrage gains are thus:
\begin{itemize}
    \item Insiders' arbitrage gain: $G_a = E[P(t) - P(t_0)|I(t_0)] = \alpha \cdot I(t_0)$
    \item Outsiders' arbitrage gain: $G_o = E[P(t') - P(t)|I(t)], \quad t' > t > t_0$
\end{itemize}
Since $I(t_0) > I(t_0 + N)$, insiders always achieve higher arbitrage gains:
\[
G_a > G_o
\]

\section{Game-Theoretical Nash Equilibrium}

Using a game-theoretic approach with strategies Ethical (E) and Unethical (U), insiders find unethical strategies dominant due to early arbitrage opportunities:
\[
U_F(U) \succ U_F(E)
\]
This dominance fosters systematic incentives for unethical behaviors.

\section{Structural Incentive for Unethical Behavior}

Isolated markets allow zero-cost token creation $C_{market} = 0$, thus insiders generate profits with negligible risk:
\[
G_a = \alpha \cdot I(t_0) - C_{market} \gg G_o = \alpha \cdot I(t_0 + N) - C_{market}, \quad t > t_0
\]

\section{Omni-Solver as Tool for Equilibrium Restoration}

The Omni-Solver provides tools to create synthetic derivatives, enabling outsiders to participate effectively and neutralize insiders' asymmetric advantages:
\[
F(t) = P(t) + \beta \cdot V(t), \quad \text{where } V(t) \text{ is fair-value premium}
\]
The availability of synthetic assets ensures equitable arbitrage opportunities:
\[
E[F(t_0)|I(t_0)] \approx E[F(t)|I(t)], \quad t \ge t_0
\]

\section{Comprehensive Equilibrium Model}

The two-sided liquidity condition ensures fairness:
\[
P_{equilibrium}(t) \approx E[P(t)], \quad \text{fair equilibrium price}
\]

\section{Conclusion}

This comprehensive derivation rigorously demonstrates:
\begin{itemize}
    \item Structural insider advantage in isolated markets
    \item Effectiveness of synthetic assets, facilitated by the Omni-Solver, in restoring fairness
    \item The necessity of financial innovation to democratize and stabilize cryptocurrency markets
\end{itemize}

\end{document} 